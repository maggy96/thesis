\section{Probleme und Herausforderungen}
\label{Message Broker:challenges}
Die Nutzung einer MOM bringt jedoch nicht nur Vorteile mit sich. Es gibt einige
Punkte, die man bei der Planung berücksichtigen sollte.

\paragraph{Sicherheit}
Zwar sind viele Vorteile mit der Kommunikation über eine zentrale Instanz
verbunden, allerdings ist damit auch jeder andere Teil der Anwendung davon
abhängig. Fällt ein Broker bzw. die gesamte MOM aus, so kann dies beispielsweise
dazu führen, dass gesendete und in der Queue vorgehaltene Nachrichten evtl.
komplett verloren gehen.
Dabei können also durch Ausfall eines einzigen Systems Daten des gesamten Systems
verloren gehen.
Ebenso muss für die Sicherheit eines weiteren Systems gesorgt werden. Kann ein
Angreifer die Kontrolle über eine MOM übernehmen, hat er potentiell Zugriff auf
alle Nachrichten, die über sie versendet werden. \cite{tanenbaum2007distributed}

\paragraph{Betrieb}
Mit einer MOM muss ein weiteres System regelmäßig gewartet und gepflegt werden.
Außerdem müssen Themen wie High Availability und Skalierung bedacht werden.
Zwar liefern die meisten Implementierungen Wege mit, diesen Themen zu begegnen,
allerdings geschieht dies selten automatisch \cite{RabbitScaling:online}.
Wenn unvorhersehbar eine große Nachrichtenmenge bei der MOM eingeht, ist
ein reibungsloser Betrieb meist trotzdem wünschenswert.
Damit Ausfälle bzw. Performanceengpässe überhaupt rechtzeitig bemerkt werden
können, muss zusätzlich ein geeignetes Monitoring betrieben werden. \cite{dobbelaere2017kafka}

\paragraph{Integration}
Jedoch ist der Betrieb nicht die einzige Herausforderung. So muss beispielsweise
für jeden Teilnehmer eine geeignete Schnittstelle zum Broker erstellt werden.
Auch wenn die meisten Broker \textit{Libraries} in verschiedenen Sprachen zu
Verfügung stellen, muss dies trotzdem erst einmal integriert werden \cite{KafkaClients:online}.
Sollte die MOM nicht bereits bei der Konzeption eines Systems eingeplant worden
sein, so entstehen Einführungskosten.
