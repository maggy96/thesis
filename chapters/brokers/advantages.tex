\label{Message Broker:advantages}
\section{Vorteile}
Durch die beschriebene Architektur einer nachrichtenorientierten Middleware
ergeben sich einige Vorteile gegenüber direktem Messaging. Neben den bereits
erwähnten sind dies die Folgenden.

\paragraph{Skalierbarkeit}
Durch die Extraktion des Messagings in eine externe Instanz kann nicht nur das
Routing effizienter, simpler und performanter geschehen, es können auch weitere
Empfänger und Sender einfacher eingebunden werden und die Netzwerktopologie wird
vereinfacht. Weiter können bei den meisten Implementierungen Redundanz
herbeigeführt und somit höhere Lasten bewältigt werden. Auf diese Weise kann beim
Erweitern der Anwendung sehr einfach die nachrichtenorientierte Middleware mit
skaliert werden \cite{curry2004message}.
\paragraph{Flexibilität}
Da die meisten Message Broker Clients für viele Programmiersprachen anbieten
und auf mehreren Plattformen betrieben werden können, wird eine große Flexibilität
erzielt. Dabei müssen die Empfänger und Sender nicht zwingend das gleiche
Protokoll verwenden \cite{dobbelaere2017kafka}.
\paragraph{Stabilität}
Gerade durch Verwendung von Queues kann eine erhöhte Stabilität erreicht werden.
Sollte ein Sender ausfallen, ist dies für die weitere asynchrone Verarbeitung
der Nachricht nicht relevant, solange diese vorher an den Broker übermittelt
wurde. Auch der Empfänger kann ausfallen, solange die Queue die Nachricht lange
genug vorhalten kann. Je nach Implementierung ist sogar der Broker ausfallsicher,
da durch Mechanismen wie Redundanz eine Unabhängigkeit von der Hardware erreicht wird \cite{dobbelaere2017kafka}.
\paragraph{Kohäsion und Kopplung}
Da bei direkter Kommunikation jeweils ein Client den anderen direkt aufruft,
sind die zwei Systeme eng miteinander verknüpft. Sollten sich Teile ändern oder
wegfallen, so müssen diese Änderungen durch das komplette Netzwerk propagiert werden.
Aufgrund dieses sehr statischen Aufbaus liegt eine hohe Kopplung vor.
Durch die Verwendung einer MOM wird die Aufgabe des Messagings komplett auf eine
eigene Instanz verschoben. Somit wird eine enge Kohäsion und eine geringe
Kopplung erzielt \cite{curry2004message}.
