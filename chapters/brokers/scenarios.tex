\label{Message Broker:scenarios}
\section{Typische Anwendungsszenarien}
Es lassen sich einige Anwendungsbeispiele zusammentragen, die besonders von der
Verwendung einer MOM profitieren.

\paragraph{Microservice-Architekturen}
Unter einem \textit{Microservice} versteht man einen unabhängigen Prozess, der
in einer \textit{Microservice-Architektur} eine einzelne bestimmte Aufgabe erfüllt.
Dabei ist eine Microservice-Architektur eine verteilte Anwendung, welche
ausschließlich aus Microservices besteht \cite{fowler2005micro}.
Es müssen also viele kleinere Anwendungen miteinander kommunizieren.
Hier kann mit einer MOM beispielsweise erreicht werden, dass nicht jeder
Microservice die Adressen seiner Kommunikationspartner kennen muss. Das
erleichtert unter anderem die Integration neuer Microservices.

\paragraph{IoT}
Auch im Bereich \textit{Internet of Things}, in dem viele leistungsschwache Geräte
permanent mit dem Internet verbunden sind, sind MOMs von Vorteil. Soll zwischen
vielen Geräten untereinander kommuniziert werden, greifen oben genannte Vorteile \cite{razzaque2016middleware}.
Da dieser Trend gerade in den letzten Jahren Einzug in viele Privathaushalte findet,
hat das Thema MOM auch eine wirtschaftliche Relevanz.
Dabei kann unter anderem von der erhöhten Skalierbarkeit der Kommunikation mit
einer MOM im Vergleich zur direkten Kommunikation profitiert werden.

\paragraph{Event Sourcing}
Event Sourcing beschreibt einen Ansatz, wie Datensätze, die häufigen Änderungen
ausgesetzt sind, programmatisch behandelt werden können. Hierbei wird für jede
Änderung ein sogenanntes \textit{Event} erstellt, welches die Änderungen zum vorherigen
Event enthält. Dabei ist es meist sinnvoll, wenn andere Teilanwendungen diese Events
erhalten oder diese zumindest zentral abgespeichert werden.
Grundsätzlich ist Event Sourcing im Kontext von Messaging nicht strikt
von Microservices bzw. allgemein aufgetrennten Anwendungen trennbar, allerdings
ist dieses Pattern in den letzten Jahren so populär geworden, dass speziell dafür
geeignete Broker existieren \cite{fowler2005event,ApacheKa84:online}.
Event Sourcing kann hier besonders von MOMs mit einer Queue, die permanente
Speicherung anbietet, profitieren, da hier alle Events abgelegt werden können.
