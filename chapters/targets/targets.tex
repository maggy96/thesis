\section{Zielsysteme}
\subsection{Anforderungen}
Wie bereits im vorhergehenden Abschnitt erarbeitet, existieren einige
Anwendungsfälle für die Verwendung einer Message-orientierten Middleware.
Dabei sind die zugrunde liegenden Anforderungen größtenteils identisch.
So haben die meisten Szenarios einen Bedarf für eine Zustellungsgarantie.
Hierbei ist \textit{at least once} aus Perspektive der Performance wohl die
am Einfachsten zu erreichende, wobei dadurch bei den Clients gefiltert werden
muss. Auch wünscht man sich in den meisten Fällen eine möglichst einfache
Skalierungsmöglichkeit und High Availability. Dies alles sind sehr generelle
Punkte, die in den meisten Fällen von der jeweiligen Software erfüllt werden.
Beispielsweise bieten RabbitMQ und Kafka beite Methoden für Skalierung und High
Availability.

Die wichtigen Anforderungen, die in einem spezifischen Anwendungsfall also
erarbeitet müssen, sind Sonderfälle, einzelne Spezielle Garantien und
Anforderungen, die nicht jeder generische Broker erfüllen kann.
Bei den vorgestellten Implementierungen Kafka und RabbitMQ ist ein
wesentlicher Unterschied beispielsweise die Speicherung von Nachrichten.
In manchen Anwendungsfällen, wie zum Beispiel bei Event Sourcing, sollten
Nachrichten (in diesem Fall Events) auch langfristig gespeichert werden.

Eine weitere praxisnahe Anforderung ist Performance. Da diese zweigeteilt
ist zwischen Latenz und Durchsatz kann man hier zwei Empfehlungen aussprechen.
Kafka bietet soweit man sich im Rahmen der Empfohlenen Einstellungen und
Verwendungsrichtlinien befindet, die größeren Durchsatzraten. RabbitMQ
jedoch bietet von Haus aus eine Kontrolle über die Latenz an - ein Feature
was Kafka nicht bieten kann. Liegt also eine Anwendung vor, in der es darauf
ankommt, Daten sehr schnell durch das System zu propagieren, statt große Mengen
davon, würde man hier eher zu RabbitMQ greifen. In den meisten Szenarien
ist die Performance von RabbitMQ auch völlig ausreichend, daher ist dieser
Vergleich zwischen Kafka und RabbitMQ eher weniger relevant.

Ebenso wurde in Tabelle \ref{featurematrix} in diesem Sinne darauf geachtet,
nur realitätsnahe Anforderungen darzustellen.
Wir werden also in dem folgenden Anwendungsszenario gezielt Anforderungen,
die nicht speziell benötigt werden bzw. von Kafka und RabbitMQ in einem
vergleichbaren Maße erfüllt werden, nicht so ausführlich behandeln, wie
die offensichtlichen Ausschlusskriterien und -anforderungen.
