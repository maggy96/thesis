\chapter{Schluss}
\label{part:end}
\section{Fazit}
In dieser Arbeit wurden die Grundlagen von Message Oriented Middlewares
beleuchtet. Dabei wurden der allgemeine Aufbau und die Funktion
einzelner Komponenten erklärt. Vor- und Nachteile einer MOM wurden
abgewägt.
Es wurden außerdem einige praxisnahe Szenarien, in denen oft eine Message
Oriented Middleware Einsatz findet, aufgezeigt. Hierbei wurde erklärt,
warum sie jeweils von dieser Verwendung profitieren.

Davon ausgehend wurde ein Eigenschaftenkatalog erarbeitet,
mithilfe dessen eine erleichterte Analyse verschiedener Implementierungen
ermöglicht wurde. Dieser umfasst die in der Praxis wichtigen Kriterien,
in denen sich Implementierungen unterscheiden können. Dabei wurde jede
dieser Eigenschaften verständlich begründet, sodass selbst fachfremde
sie verwenden können, um eine Auswahl zu treffen.

Anschließend wurden die verbreiteten Implementierungen analysiert und
miteinander auf Grundlage dieses Eigenschaftenkatalogs verglichen.
Zusätzlich wurde eine grundsätzlich verschiedene Neuentwicklung analysiert.
Hierdurch wurde gezeigt, dass der Katalog anwendbar ist
und auch für zukünftige Implementierungen genutzt werden kann.

Zur Validierung des Eigenschaftenkatalogs wurde dieser abschließend auf ein
praktisches Szenario angewendet. Eine Integration in den durch die CHECK24
Versicherungsservice GmbH gegebenen Anwendungsfall wurde geplant und in
allen Punkten des Katalogs untersucht. Anschließend wurde eine Empfehlung
ausgesprochen. Diese wurde auch vom Rest des Teams nach gemeinsamer Analyse
getragen.

Die Arbeit informiert ausführlich über die führenden Implementierungen,
gibt jedoch gleichzeitig ein Werkzeug mit an die Hand, mit dem Neuentwicklungen
eingeordnet werden können. Wie demonstriert wurde, kann sie als Grundlage
für Entscheidungen in echten Anwendungsfällen dienen.

\section{Ausblick}
Eine Message Oriented Middleware findet in vielen modernen Systemen Anwendung.
Eine sauber getrennte, asynchrone, und vor allem performante
Kommunikationslösung für Softwaresysteme wird auch in nächster Zukunft aus den
meisten Systemen nicht wegzudenken sein.
Dabei wächst das Angebot an Lösungen stetig. Die Spezialisierung der einzelnen
Lösungen ist dabei hoch, die meisten Neuentwicklungen sind auf einen bestimmten
Anwendungsfall zugeschnitten.
Somit wäre es sicher interessant, den Vergleich dieser Arbeit auf weitere
Implementierungen auszuweiten. Diese werden sich jedoch untereinander weniger
stark unterscheiden als Kafka und RabbitMQ.

Auch die neueste Entwicklung - Brokerlose MOMs - könnten mit den erarbeiteten
Metriken leicht eingeschätzt werden, auch wenn ihre detaillierte Behandlung
den Rahmen der Arbeit gesprengt hätte.
Diese Entwicklung lohnt es sich jedoch in jedem Fall zu verfolgen.
Dabei gibt es andere Mechanismen zu erforschen und optimieren, wie beispielsweise
die zwingend notwendige Service Discovery, mit der solche Peer to Peer Systeme
funktionieren. Da diese Entwicklungen noch relativ neu sind, wird hier in den
nächsten Jahren noch viel weiterentwickelt werden.

RabbitMQ sticht mit seinen Enterprise Eigenschaften hervor, aufgrund derer es
trotz vergleichsweise niedrigem Durchsatz immer noch große Verbreitung findet.
Ob RabbitMQ in naher Zukunft von einer anderen Implementierung abgelöst werden
kann, die einen ähnlichen Support für viele Protokolle anbietet, bleibt
spannend. Ebenso wäre eine Entwicklung interessant, die neue Architekturen,
wie Peer to Peer MOMs, in Verbindung mit dem Support für standardisierte
Protokolle bringt. Dies wäre beispielsweise für den IoT-Bereich relevant.

Es lässt sich abschließend sagen, dass in den nächsten Jahren sicher noch viel
auf dem Gebiet der Message Oriented Middleware passieren wird. Diese Arbeit hat
eine Übersicht über den aktuellen Stand der Entwicklung gegeben, während die
verwendeten Methoden auch in der Zukunft verwendet werden können.
