%%%%%%%%%%%%%%%%%%%%%%%%%%%%%%%%%%%%%%%%%%%%%%%%%%%%%%%%%%%%%%%%%%%%%%%%%%%%%%%%
%%%%%%%%%%%%  Introduction
%%%%%%%%%%%%%%%%%%%%%%%%%%%%%%%%%%%%%%%%%%%%%%%%%%%%%%%%%%%%%%%%%%%%%%%%%%%%%%%%
\section{Problemstellung}
\label{intro:problem}
Um einen möglichst universellen Vergleich zwischen Message Oriented Mid\-dlewares
zu ermöglichen, muss der Begriff zuerst umfangreich definiert werden.
Nur auf dieser Grundlage ist eine Gegenüberstellung von Implementierungen sinnvoll.

Weiter soll der Vergleich aufgrund ausgewählter, wichtiger Eigenschaften
von Message Oriented Middlewares geschehen.
Diese müssen also erarbeitet und anhand Ihrer Relevanz eingeschätzt werden.
Dabei können Eigenschaften beliebig detailliert oder abstrakt sein.
Es besteht also eine Herausforderung da\-rin, ein geeignetes Abstraktionslevel zu
finden. Anschließend soll dieser Eigenschaftenkatalog anhand eines Praxisteils
evaluiert werden.

Eine weitere Herausforderung besteht im Vergleich bestehender Message Oriented
Middlewares. Es sollte aus den verbreiteten Lösungen eine repräsentative Menge
ausgewählt werden. Also eine Menge, die einerseits groß genug ist, um genug
verschiedene Repräsentanten zu beinhalten, andererseits aber klein genug ist,
um noch übersichtlich und relevant zu sein. 

Somit sollten die Vergleichspunkte fundiert erarbeitet und nach Relevanz
bewertet werden. Letztendlich sollte der Anspruch sein, eine realistisch nutzbare
Orientierung bei der Planung eines Systems zu erstellen.
Dementsprechend sind die grundlegenden Herausforderungen dieser Arbeit die
Erarbeitung eines Eigenschaftenkatalogs und die Analyse einzelner ausgewählter
Implementierungen damit, sowie die Anwendung und Evaluierung anhand eines
Praxisbeispiels.
