%%%%%%%%%%%%%%%%%%%%%%%%%%%%%%%%%%%%%%%%%%%%%%%%%%%%%%%%%%%%%%%%%%%%%%%%%%%%%%%%
%%%%%%%%%%%%  Abstract
%%%%%%%%%%%%%%%%%%%%%%%%%%%%%%%%%%%%%%%%%%%%%%%%%%%%%%%%%%%%%%%%%%%%%%%%%%%%%%%%
\chapter*{Kurzbeschreibung}
% was, warum?
In der modernen Softwareentwicklung finden sich in großen Anwendungen häufig
sogenannten \textit{Message Oriented Middlewares} wieder. Hierbei handelt es sich
um Systeme, die eine effiziente Kommunikation zwischen einer großen Anzahl
von Prozessen ermöglichen, oft auch in verteilten Systemen oder heterogenen
Netzen. In den letzten Jahren gab es vermehrt Weiterentwicklungen im Bereich
der Message Oriented Middleware, da diese nicht zuletzt auch in Microservices
vermehrte Verbreitung gefunden haben.
Hierbei handelt es sich um Message Oriented Middlewares, die als eigene
Anwendung unabhängig vom restlichen System betrieben werden.
Es existieren viele Implementierungen, wobei sich diese untereinander
stark unterscheiden können.
Bei der Konzeption eines Systems, welches eine Message Oriented Middleware verwendet, 
verliert man daher schnell den Überblick über diese.

% arbeit
Aus diesem Grund wird diese Arbeit einen Überblick über bestehende Lösungen geben und diese
auf konzeptioneller Ebene vergleichen. Einzelne Punkte, in denen sich diese
unterscheiden, werden dabei begründet erarbeitet und zu einem Katalog
zusammengefasst werden. 
Dies ermöglicht, die Erkenntnisse aus der Arbeit
auch auf zukünftige Entwicklungen anzuwenden.
Daraufhin wird ein Vergleich erstellt, um die aktuell führenden Implementierungen aufzeigen
und ihre Unterschiede in den Punkten dieses Kataloges gegenüberzustellen. 
Letztendlich soll für ein Zielsystem des Betreuers
(CHECK24 Versicherungsservice GmbH) eine geeignete Implementierung
gewählt werden. Da die Umsetzung den Rahmen der Arbeit sprengen würde, wird
sich hier auf eine Analyse der möglichst optimalen Lösungen beschränkt.
Hierzu werden für dieses System spezifische Anforderungen identifiziert und
versucht, mit dem im zweiten Teil der Arbeit vorgestellten Katalog eine 
Kategorie für dieses Zielsystem zu finden. Daraufhin kann eine Empfehlung einer Lösung ausgesprochen
werden. Da eine konkrete Implementierung entfällt, werden theoretische Probleme
der empfohlenen Implementierung erörtert und Lösungsvorschläge präsentiert werden.
% Wozu -> Forschungsfragen

Zusammenfassend wird zum einen ein Überblick über Implementierungen erstellt,
der diese konkret vergleicht. Dabei werden einzelne Punkte des Vergleichs vorher
erarbeitet und anschließend nach Relevanz eingeordnet. Letztendlich werden
anhand eines Fallbeispiels die vorherigen Abschnitte validiert und ein Fazit
gezogen.
