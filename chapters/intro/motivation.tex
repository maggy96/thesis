%%%%%%%%%%%%%%%%%%%%%%%%%%%%%%%%%%%%%%%%%%%%%%%%%%%%%%%%%%%%%%%%%%%%%%%%%%%%%%%%
%%%%%%%%%%%%  Acknowledgements
%%%%%%%%%%%%%%%%%%%%%%%%%%%%%%%%%%%%%%%%%%%%%%%%%%%%%%%%%%%%%%%%%%%%%%%%%%%%%%%%
\section{Motivation}
\label{intro:motivation}
In der CHECK24 Versicherungsservice GmbH kommen im 
\textit{Versicherungscenter}\footnote{https://www.check24.de/versicherungscenter/,
abgerufen am 23.11.2018} im Laufe des Jahres neue Businessanforderungen auf die
vorhandene Infrastruktur zu. Im Rahmen der Neuerungen wurde erwogen, eine
Message Oriented Middleware in das System zu integrieren. Dabei reichten jedoch
die Erfahrungen anderer Teams nicht aus, um eine Entscheidung bezüglich der
konkreten Implementierungen zu fällen.

Die konkreten Anforderungen waren zu unterschiedlich zu denen von anderen Teams, um
von Ihren Erfahrungen sinnvolle Schlüsse ableiten zu können. Daher soll eine
Evaluierung der Optionen stattfinden, was den Anreiz für diese Arbeit gab. Es soll
also nicht nur eine Analyse und ein Vergleich von marktführenden
Lösungen stattfinden, sondern auch in einem Praxisteil das konkrete Szenario
untersucht und eine Empfehlung ausgesprochen werden.

Der bis dahin aufgestellte Eigenschaftenkatalog und die Analyse existierender
Implementierungen sollen helfen, eine fundierte Wahl einer Lösung zu treffen. Zudem
sollen die erarbeiteten Grundlagen Missverständnisse vorbeugen und zum Verständnis
bei der Integration beitragen.
