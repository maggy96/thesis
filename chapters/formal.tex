\section{Eigenschaften einer nachrichtenorientierten Middleware}
\label{formal}

\begin{table}[h]
  \centering
  \begin{tabular}{|l|l|}
    \hline
    \textbf{Eigenschaft} & \textbf{Optionen} \\ \hline
    Standardisierte Protokolle & AMQP, MQPP, - \\ \hline
    Clientbibliotheken & siehe \ref{compability:table}\\ \hline
    Möglichkeit zur Skalierung & Replikation, - \\ \hline
    Durchsatz & hoch, niedrig \\ \hline
    Latenz & stabil, nicht stabil \\ \hline
    Routingoptionen & Topic, Channel, Content, Statisch \\ \hline
    Zustellgarantien & at most once, at least once, exactly once \\ \hline
    Speicherort & RAM, Festplatte \\ \hline
    Langzeitspeicherung & unterstützt, nicht unterstützt \\ \hline
    Ordnungsgarantien & total order, partitioned order, no order \\ \hline

  \end{tabular}
  \caption{Eigenschaftenkatalog}
  \label{catalogue}
\end{table}

Nach den Erkenntnissen des vorherigen Abschnitts können nun Eigenschaften
einer MOM erarbeitet werden, die besonders zur Unterscheidung von
Implementierungen geeignet sind. Hierbei wird anhand des im Abschnitt
\ref{definition:definition} beschriebenen Aufbaus vorgegangen.
Die erarbeiteten Eigenschaften sind in der Tabelle \ref{catalogue} zusammen mit
den für die jeweiligen Eigenschaften verfügbaren Optionen dargestellt. Im
Folgenden werden die einzelnen Punkte erklärt.
Anschließend werden sie den folgenden Abschnitten dazu verwendet werden,
einzelne Implementierungen zu analysieren und miteinander zu vergleichen.

\subsection{Sender und Empfänger}
Wie in den Grundlagen erwähnt, sind Producer und Consumer unabhängige Instanzen.
Es ist jedoch nicht ausgeschlossen, dass Consumer gleichzeitig Producer sind und
umgekehrt. Somit fließt in die Kriterien für die Auswahl von Brokern ein, welche
Clients dieser unterstützen kann. 
Dabei muss sowohl betrachtet werden, ob \textbf{standardisierte Protokolle}
verwendet werden, als auch, wie leicht Clients in verbreiteten Sprachen
implementiert werden können.
Nach Statistiken von Github sind die 2018 am meisten verwendeten Sprachen in
absteigender Reihenfolge Javascript, Java, Python, PHP, C++, C\#, Typescript,
Shell, C und Ruby \cite{Octoverse:online}.
Da Typescript auch Javascript Pakete verwenden kann und somit
in diesem Kontext die Unterscheidung zwischen Javascript und Typescript
irrelevant ist, sowie Shell für die meisten großen Softwarearchitekturen nicht
infrage kommt, werden nur die anderen acht Sprachen zur Bewertung herangezogen. Dabei entfällt
eine Bewertung der unterstützten Plattformen, da diese eher von den
\textbf{Clientbibliotheken} bzw. den unterstützten Programmiersprachen abhängen.
Diese sind in Tabelle \ref{compability:table} dargestellt.

In einer Implementierung ist selbstverständlich auch die Zahl an
gleichzeitig unterstützten Clients nicht unlimitiert. Sie hängt von mehreren
Faktoren ab, allerdings kann man hierbei durchaus an die Grenzen einer
Hardware stoßen.
In den meisten Fällen wird daher Replikation angeboten
\cite{ApacheKa84:online,RabbitMQ:online}. Das bedeutet, dass
einige oder alle Daten der MOM in mehreren, unabhängigen Instanzen gespeichert
und synchronisiert werden. Somit werden Implementierungen auf \textbf{Möglichkeit
zur Skalierung} untersucht.

Ebenso limitiert ist eine Implementierung darin, wie viele Nachrichten sie
gleichzeitig versenden und empfangen kann, da hier Faktoren wie
Hardwareressourcen und Netzwerk eine Rolle spielen.
Relevante Metriken sind hierbei zum einen der \textbf{Durchsatz}, also die Menge
von Nachrichten, die eine MOM pro Zeiteinheit verarbeiten kann, und zum anderen
die \textbf{Latenz}, also die Versanddauer einer Nachricht. Diese Eigenschaften
können beispielsweise in Echtzeitsystemen harte Anforderungen sein.

\subsection{Routing}
Wie bereits in Abschnitt \ref{definition:routing} erwähnt, unterscheidet
man neben statischem Routing zwischen \textit{Channel-}, \textit{Topic-} und
\textit{Content-} basiertem Routing. Daher werden diese \textbf{Routingoptionen}
in den Eigenschaftenkatalog aufgenommen.

Beim Versand können jedoch Netzwerkpakete verloren gehen, Clients oder Broker
ausfallen, oder Antwortpakete verloren gehen, wodurch die Nachricht mehrfach
oder gar nicht gesendet wird. Daher sollen Implementierungen auf
\textbf{Zustellgarantien} untersucht werden.

Bei \textbf{at most once} wird eine Nachricht maximal einmal zugestellt.
Hierbei wird nicht berücksichtigt, ob eine Nachricht z.b. durch
Paketverlust verloren geht. Eine weitere Variante ist
\textbf{at least once}. Hierbei wird jede Nachricht garantiert zugestellt,
allerdings kann es zu Mehrfachzustellungen kommen und es werden mehr Ressourcen
benötigt. Die dritte Alternative \textbf{exactly once}, bei dem
keine Nachricht verloren geht und auch keine Nachricht mehr als einmal zugestellt wird,
ist von allen Varianten die ressourcenintensivste. \cite{dobbelaere2017kafka}

\subsection{Queues}
Da nicht jede Implementierung eine Queue verwendet, spielt diese nur bei Verwendung
eine Rolle. Jedoch sind hier auch wieder Limitierungen von Hardware anzutreffen.
So kann eine Queue nicht  unendlich viele Nachrichten speichern, da
meist begrenzter Speicherplatz, insbesondere bei flüchtigen Speichermedien vorliegt.
Werden Nachrichten beispielsweise nach dem Versand direkt wieder gelöscht, so wird
natürlich weniger Speicher benötigt.
Der \textbf{Speicherort} (etwa auf einer Festplatte oder im Arbeitsspeicher),
und die Option zur \textbf{Langzeitspeicherung} sind also interessant.

Da bei der Verwendung von Queues alle Nachrichten diese durchlaufen, ist
außerdem relevant, ob die Reihenfolge dieser dabei erhalten wird. Man spricht
in diesem Kontext von \textbf{Ordnungsgarantien}. Dabei kann eine MOM
\textit{no order}, also keine Ordnungsgarantien, \textit{partitioned order}, also eine 
Ordnungsgarantie über eine Teilmenge von Nachrichten, oder \textit{total order},
also eine absolute Ordnung, anbieten. \cite{dobbelaere2017kafka}

Über die erarbeiteten Punkte hinaus ist es schwer, Eigenschaften zu finden,
in denen sich Implementierungen wesentlich unterscheiden. Die meisten aktuellen
Implementierungen sind auf einen bestimmten Anwendungsfall ausgerichtet, was den
Vergleich außerhalb genannter Eigenschaften erschwert \cite{kleppmann2017designing}.

\begin{figure}[]
	\begin{tabular}{l|llllllll}
		Broker/Sprache                         & Javascript & Java & Python & PHP  & C++  & C\#  & C    & Ruby \\
		\cline{1-9}
		Kafka \cite{KafkaClients:online}       & (ja)       & ja   & (ja)   & (ja) & (ja) & (ja) & (ja) & (ja) \\
		RabbitMQ \cite{RabbitMQClients:online} & (ja)       & ja   & (ja)   & (ja) & (ja) & ja   & ja   & (ja) \\
		NSQ \cite{NSQ:online}                  & ja         & (ja) & ja     & (ja) & (ja) & ja   & (ja) & (ja) \\
	\end{tabular}
	\caption{Kompatibilitätsmatrix Message Broker. Drittanbieterbibliotheken in Klammern, soweit vom Hersteller empfohlen}
	\label{compability:table}
\end{figure}