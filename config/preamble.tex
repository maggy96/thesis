% ------------------------------------------------------------------
% DOKUMENTEINSTELLUNGEN
% ------------------------------------------------------------------
\documentclass[
paper=a4,           % Page dimensions
fontsize=12pt,      % Schriftgröße 12 Punkt
DIV=12,             % Teilungsfaktor für Satzspiegel
listof=totoc,       % Tabellen & Abbildungsverzeichnis ins Inhaltsverzeichnis
bibliography=totoc, % Literaturverzeichnis ins Inhaltsverzeichnis
appendixprefix,   	% "Appendix" Prefix für Anhang
headinclude,        % Kopfzeile mit in Satzspiegel aufnehmen
twoside,            % Page margins for two-sided printing
open=right,         % Begin new chapters on a right-hand side
parskip,            % Space after last line of paragraph: 1em
draft=false,        % Draft version?
titlepage=on,       % Title at own page
fleqn,              % Formulas left-aligned
]{scrbook}          % oder {scrreprt}

% Alle Überschriften fett und *mit* Serifen (normale Schriftart und Größe)
\setkomafont{chapter}{\normalfont\bfseries\huge}
\setkomafont{section}{\normalfont\bfseries\Large}
\setkomafont{subsection}{\normalfont\bfseries\large}
\setkomafont{subsubsection}{\normalfont\bfseries}

% \usepackage{morewrites}
\usepackage{scrwfile}

% Repair some problems of komascript with other packages
\usepackage{scrhack}

% Custom page geometry
\usepackage{geometry}

\renewcommand{\textfraction}{0.01} % Minimum fraction of page for text
\renewcommand{\topfraction}{0.99} % Fraction of page from top that may contain figures
\renewcommand{\bottomfraction}{0.99} % Fraction of page from bottom that may contain figures
\renewcommand{\floatpagefraction}{0.99} % Avoid separate pages that only contain figures

% Page break before each section
\let\oldsection = \section
\renewcommand{\section}[1]{
  \clearpage
  \oldsection{#1}
}

% unterschiedliche linespaces (für die comments)
\usepackage{setspace}

% Handle page numbers on empty pages
\makeatletter
\def\cleardoublepage{\clearpage\if@twoside \ifodd\c@page\else
  \hbox{}
  \thispagestyle{empty} % no page number at empty pages
  % \thispagestyle{plain} % page number at empty pages
  \newpage
  \if@twocolumn\hbox{}\newpage\fi\fi\fi}
\makeatother

% ------------------------------------------------------------------
% ERWEITERUNGEN: SPRACHEINSTELLUNGEN
% ------------------------------------------------------------------

% Englische Spracheinstellung
% \usepackage[ngerman]{babel}
% Deutsche Spracheinstellung
\usepackage[ngerman]{babel}

% Kodierung
\usepackage[T1]{fontenc}
\usepackage[utf8]{inputenc}

% ------------------------------------------------------------------
% ERWEITERUNGEN: ALLGEMEIN
% ------------------------------------------------------------------

% Packages to allow inclusion of graphics
\usepackage{graphicx}
\usepackage{subcaption}
\usepackage{float}
\usepackage{caption}

% For creating coloured text and background
\usepackage[svgnames,rgb]{xcolor}

% Unformatierten Text einfügen
\usepackage{verbatim}

\usepackage{url}

% Einträge in PDF verlinken, keine Ränder
\usepackage{hyperref}

\hypersetup{
% bookmarks=true,                              % show bookmarks bar?
  bookmarksnumbered=true,
  pdftitle={Vergleich aktueller Implementierungen von Message Oriented Middlewares},                % title
  pdfauthor={Magnus Görlitz},                % author
  pdfsubject={Thesis},                   % subject of the document
  pdfkeywords={Modeling} {PubSub} {Message Broker}, % list of keywords
  colorlinks=true,                             % false: boxed links; true: colored links
  linkcolor=black,                             % color of internal links
  citecolor=red,                               % color of links to bibliography
  pdfborder={0 0 1}
}

% Provides a solution to the problem with hyperref that links
% to floats actually anchor to the place below the float's caption,
% instead of anchoring to the beginning of the float
\usepackage[all]{hypcap}

% EPS Grafiken einbinden
\usepackage{epstopdf}

% Decorative chapter headings | Muss nach xcolor geladen werden!
\usepackage[grey, utopia]{quotchap}

\usepackage{enumitem}

\usepackage{algorithm}
\usepackage[noend]{algpseudocode}

\usepackage{wrapfig} % Bilder mit Fließtext

\usepackage{tikz}


% ------------------------------------------------------------------
% ERWEITERUNGEN: ZITATE UND REFERENZEN
% ------------------------------------------------------------------

% Context sensitive quotation facilities
% Required by biblatex for multilingual quoting
\usepackage{csquotes}

% Use biblatex instead of bibtex -- it's the future.
\usepackage[
backend=biber,
natbib=true,        % Provide aliases for natbib's citation commands
citestyle=numeric,
style=numeric
]{biblatex}

\addbibresource{bibliography/references.bib}

% ------------------------------------------------------------------
% ERWEITERUNGEN: MATH
% ------------------------------------------------------------------

% Typical maths resource packages
\usepackage{amsmath,amssymb,amsfonts}

% Intelligent cross-referencing
\usepackage{cleveref}

% Mathematical typesetting with the Palatino fonts
\usepackage{mathpazo}

% Capital handwriting letters for mathematical purposes
% \usepackage{mathrsfs}

% Mathematical symbol font that contains symbol for disjoint union
\usepackage{MnSymbol}

% AMS theorems
\usepackage{amsthm}

\usepackage{framed}
% \usepackage[framed]{ntheorem}

% darstellung mathematischer definitionen
\usepackage{shadethm}


% ------------------------------------------------------------------
% ERWEITERUNGEN: TABLE
% ------------------------------------------------------------------

% Tables with variable column width
\usepackage{tabularx}

% Additional table commands for academic publications
\usepackage{booktabs}

% Rotate objects (used for vertical text in table cells)
\usepackage{rotating}

% Multilined table cells
\usepackage{makecell}

% Create tabular cells spanning multiple rows
\usepackage{multirow}

% Tables that span over several pages
\usepackage{longtable}

% ------------------------------------------------------------------
% ERWEITERUNGEN: VERZEICHNISSE
% ------------------------------------------------------------------

% Code-Listings
\usepackage{listings}
\lstset{
	basicstyle=\ttfamily\tiny,
	tabsize=2,
	breaklines=true,
	frame=single,
	captionpos=b,
	showstringspaces=false,
	numbers=left,
	numberfirstline=false,
	firstnumber=1,
	stepnumber=2,
	numbersep=5pt
  }
  
% Glossary
\usepackage[toc]{glossaries}
\makeglossaries
\loadglsentries{glossaries}

% ------------------------------------------------------------------
% KOPF-/FUSSZEILE
% ------------------------------------------------------------------
\renewcommand{\ttdefault}{lmtt} % bold texttt

% Use KOMA-recommended package for header/footer style setup
\usepackage[%
headsepline, % Trennlinie zwischen Inhalt und Kopfzeile
footsepline, % Trennlinie zwischen Inhalt und Fußzeile
automark     % Automatischer Inhalt von Seitenüberschriften
]{scrlayer-scrpage}

\usepackage{textcase} % for \MakeTextLowercase

\clearpairofpagestyles
\lehead{\MakeTextLowercase{\textsc{\leftmark}}}
\rohead{\MakeTextLowercase{\textsc{\rightmark}}}
\ofoot{\pagemark}

% Abstand zwischen Kopf-/Fußzeile und Trennlinie
\chead{\rule[-\dimexpr \dp\strutbox+1ex\relax]{0pt}{\baselineskip}}
\cfoot{\rule[-\dimexpr \dp\strutbox+1ex\relax]{0pt}{\baselineskip}}

\addtokomafont{pageheadfoot}{\small\textcolor{gray!85}}
\setkomafont{pagenumber}{\small\textcolor{gray!85}}

% Separator line for headers/footers
\KOMAoptions{headsepline=.4pt:\textwidth}
\KOMAoptions{footsepline=.4pt:\textwidth}

\setkomafont{headsepline}{\color{gray!75}}
\setkomafont{footsepline}{\color{gray!75}}


% beamer-style shaded text box with title
\usepackage{tcolorbox}
\usetikzlibrary{shadings,shadows}

\newenvironment{frameblock}[1]{%
  \tcolorbox[beamer,%
  noparskip,breakable,
  colback=LightGray,colframe=DarkGray,%
  colbacklower=DarkBlue!75!LightBlue,%
  title=#1]}%
{\endtcolorbox}

\newenvironment{frameblockk}[1]{%
  \tcolorbox[beamer,%
  noparskip,breakable,width=0.7\textwidth,
  colback=LightGray,colframe=DarkGray,%
  colbacklower=DarkBlue!75!LightBlue,%
  title=#1]}%
{\endtcolorbox}
\usepackage{fancybox}
% ------------------------------------------------------------------
% KOMMANDOS
% ------------------------------------------------------------------
% Informationen ------------------------------------------------------------
% 	Definition von globalen Parametern, die im gesamten Dokument verwendet
% 	werden können (z.B auf dem Deckblatt etc.).
% --------------------------------------------------------------------------

% Allgemeine Informationen
\title{Vergleich aktueller Implementierungen von Message Oriented Middlewares}
% \subtitle{and its applications}
\author{Magnus Görlitz}
\newcommand{\place}{Augsburg}

\newcommand{\art}{Thesis}
\newcommand{\degreetext}{for the degree of}
\newcommand{\degree}{Bachelor of Science (B.Sc.)}
\newcommand{\universitaet}{University of Augsburg}
\newcommand{\institut}{Department of Computer Science}
\newcommand{\professur}{Software Methodologies for Distributed Systems}

\newcommand{\monat}{February}
\newcommand{\jahr}{2019}

\newcommand{\monatV}{*MonthDefense*}
\newcommand{\jahrV}{2019}

\newcommand{\gutachtereins}{\textbf{Prof.~Dr.~Bernhard Bauer}, Department of Computer Science,}
\newcommand{\gutachterzwei}{\textbf{Prof.~Dr.~Bernhard Möller}, Department of Computer Science,}
\newcommand{\gutachterinst}{University of Augsburg, Germany}
%\newcommand{\gutachterdrei}{Prof.~Dr.~Severus Snape (University of Hogwarts)}

% Farben
\definecolor{lightgray}{gray}{0.95}
\definecolor{darkgray}{gray}{0.3}


% Abkürzungen
\newcommand{\vgl}{Vgl.\ }
\newcommand{\ua}{\mbox{u.\,a.\ }}
\newcommand{\zB}{\mbox{z.\,B.\ }}
%%% Local Variables:
%%% mode: latex
%%% TeX-master: "../TemplateSMDS"
%%% End:

% --------------------------------
% Neue global benutzbare Kommandos
% --------------------------------
\newcommand\insertemptypage{%
  \newpage
  \thispagestyle{empty}
  \mbox{}
}

% description in enum lists
\newcommand\litem[1]{\item{\bfseries #1\\}}

% =======================================================================
% TIKZ
% =======================================================================

\newcommand*\circled[1]{\tikz[baseline=(char.base)]{%
    \node[shape=circle,draw,minimum size=4mm,inner sep=0pt] (char) {#1};}}
\providecommand{\csnode}[1]{\circled{#1}}

\newcommand*\circledf[1]{\tikz[baseline=(char.base)]{%
    \node[shape=circle,draw,fill=black,minimum size=5mm,inner sep=0pt] (char) {#1};}}
\providecommand{\csnodef}[1]{\circledf{\color{white} #1}}


\newcommand{\csrect}[2][]{%
  \tikz[baseline=(char.base)] \node [shape=rectangle, draw=black, inner sep=1pt, rounded corners, minimum size=4mm,#1] {#2};%
}

% =======================================================================
% Layout / Struktur
% =======================================================================

\newcommand{\marginlabel}[1]{\mbox{}\marginline{\hspace{0pt}\RaggedRight\footnotesize #1}}

% Neue Zeile erzwingen (z.B. für 'description' Umgebung)
\newcommand\fnl{\textcolor{white}{\\}}

% Kleine Überschrift
\newcommand\subpar[1]{\vspace{15pt} \textbf{#1}\vspace{10pt} \\}

% Zentrierte, farbig hinterlegte Textbox
\newcommand\textbox[1]{%
  \vspace*{2mm}
  \begin{center}
    \fcolorbox{darkgray}{lightgray}
    {
      \parbox{0.80\textwidth}
      {
        \begin{flushleft}
          #1
        \end{flushleft}
      }
    }
  \end{center}
  \vspace*{2mm}
}

% Nicht zentrierte, farbig hinterlegte Textbox
\newcommand\textboxnc[1]{%
  \vspace*{2mm}
  \fcolorbox{darkgray}{lightgray}
  {
    \parbox{0.60\textwidth}
    {
      \begin{flushleft}
        #1
      \end{flushleft}
    }
  }
  \vspace*{2mm}
}

% Nicht zentrierte, farbig hinterlegte Textbox
\newcommand\textboxm[2]{%
  \vspace*{2mm}
  \fcolorbox{darkgray}{lightgray}
  {
    \parbox{#1\textwidth}
    {
      \begin{flushleft}
        #2
      \end{flushleft}
    }
  }
  \vspace*{2mm}
}

% Nicht zentrierte, farbig hinterlegte Textbox
\newcommand\textboxdfa[1]{%
  \vspace*{2mm}
  \fcolorbox{darkgray}{lightgray}
  {
    \parbox{0.45\textwidth}
    {
      \begin{flushleft}
        #1
      \end{flushleft}
    }
  }
  \vspace*{2mm}
}

\makeatletter
\newcommand{\paragr}[1]{\@startsection{paragraph}{4}{\z@}%
  {-3.25ex\@plus -1ex \@minus -.2ex}%
  {1.5ex \@plus .2ex}
  {\normalfont\normalsize\bfseries}}
\makeatother

% =======================================================================
% Mathematik
% =======================================================================

\providecommand{\abs}[1]{\lvert#1\rvert}
\providecommand{\Abs}[1]{\left\lvert#1\right\rvert}
\providecommand{\norm}[1]{\left\Vert#1\right\Vert}
\providecommand{\Trace}[1]{\ensuremath{\Tr\{\,#1\,\}}} % Trace /Spur

% -- Sets of numbers -- 
\newcommand{\nat}{\mathbb{N}}     % Integers
\newcommand{\rat}{\mathbb{Q}}     % Rationals
\newcommand{\complex}{\mathbb{C}} % Complex numbers
\newcommand{\real}{\mathbb{R}}    % Real numbers


\RequirePackage[ngerman=ngerman-x-latest]{hyphsubst}
\usepackage[graphicx]{realboxes}
